\documentclass[]{article}
\usepackage{lmodern}
\usepackage{amssymb,amsmath}
\usepackage{ifxetex,ifluatex}
\usepackage{fixltx2e} % provides \textsubscript
\ifnum 0\ifxetex 1\fi\ifluatex 1\fi=0 % if pdftex
  \usepackage[T1]{fontenc}
  \usepackage[utf8]{inputenc}
\else % if luatex or xelatex
  \ifxetex
    \usepackage{mathspec}
  \else
    \usepackage{fontspec}
  \fi
  \defaultfontfeatures{Ligatures=TeX,Scale=MatchLowercase}
\fi
% use upquote if available, for straight quotes in verbatim environments
\IfFileExists{upquote.sty}{\usepackage{upquote}}{}
% use microtype if available
\IfFileExists{microtype.sty}{%
\usepackage{microtype}
\UseMicrotypeSet[protrusion]{basicmath} % disable protrusion for tt fonts
}{}
\usepackage[margin=1in]{geometry}
\usepackage{hyperref}
\hypersetup{unicode=true,
            pdftitle={Taller 1 Multivariado},
            pdfauthor={Diego Torres},
            pdfborder={0 0 0},
            breaklinks=true}
\urlstyle{same}  % don't use monospace font for urls
\usepackage{color}
\usepackage{fancyvrb}
\newcommand{\VerbBar}{|}
\newcommand{\VERB}{\Verb[commandchars=\\\{\}]}
\DefineVerbatimEnvironment{Highlighting}{Verbatim}{commandchars=\\\{\}}
% Add ',fontsize=\small' for more characters per line
\usepackage{framed}
\definecolor{shadecolor}{RGB}{248,248,248}
\newenvironment{Shaded}{\begin{snugshade}}{\end{snugshade}}
\newcommand{\KeywordTok}[1]{\textcolor[rgb]{0.13,0.29,0.53}{\textbf{#1}}}
\newcommand{\DataTypeTok}[1]{\textcolor[rgb]{0.13,0.29,0.53}{#1}}
\newcommand{\DecValTok}[1]{\textcolor[rgb]{0.00,0.00,0.81}{#1}}
\newcommand{\BaseNTok}[1]{\textcolor[rgb]{0.00,0.00,0.81}{#1}}
\newcommand{\FloatTok}[1]{\textcolor[rgb]{0.00,0.00,0.81}{#1}}
\newcommand{\ConstantTok}[1]{\textcolor[rgb]{0.00,0.00,0.00}{#1}}
\newcommand{\CharTok}[1]{\textcolor[rgb]{0.31,0.60,0.02}{#1}}
\newcommand{\SpecialCharTok}[1]{\textcolor[rgb]{0.00,0.00,0.00}{#1}}
\newcommand{\StringTok}[1]{\textcolor[rgb]{0.31,0.60,0.02}{#1}}
\newcommand{\VerbatimStringTok}[1]{\textcolor[rgb]{0.31,0.60,0.02}{#1}}
\newcommand{\SpecialStringTok}[1]{\textcolor[rgb]{0.31,0.60,0.02}{#1}}
\newcommand{\ImportTok}[1]{#1}
\newcommand{\CommentTok}[1]{\textcolor[rgb]{0.56,0.35,0.01}{\textit{#1}}}
\newcommand{\DocumentationTok}[1]{\textcolor[rgb]{0.56,0.35,0.01}{\textbf{\textit{#1}}}}
\newcommand{\AnnotationTok}[1]{\textcolor[rgb]{0.56,0.35,0.01}{\textbf{\textit{#1}}}}
\newcommand{\CommentVarTok}[1]{\textcolor[rgb]{0.56,0.35,0.01}{\textbf{\textit{#1}}}}
\newcommand{\OtherTok}[1]{\textcolor[rgb]{0.56,0.35,0.01}{#1}}
\newcommand{\FunctionTok}[1]{\textcolor[rgb]{0.00,0.00,0.00}{#1}}
\newcommand{\VariableTok}[1]{\textcolor[rgb]{0.00,0.00,0.00}{#1}}
\newcommand{\ControlFlowTok}[1]{\textcolor[rgb]{0.13,0.29,0.53}{\textbf{#1}}}
\newcommand{\OperatorTok}[1]{\textcolor[rgb]{0.81,0.36,0.00}{\textbf{#1}}}
\newcommand{\BuiltInTok}[1]{#1}
\newcommand{\ExtensionTok}[1]{#1}
\newcommand{\PreprocessorTok}[1]{\textcolor[rgb]{0.56,0.35,0.01}{\textit{#1}}}
\newcommand{\AttributeTok}[1]{\textcolor[rgb]{0.77,0.63,0.00}{#1}}
\newcommand{\RegionMarkerTok}[1]{#1}
\newcommand{\InformationTok}[1]{\textcolor[rgb]{0.56,0.35,0.01}{\textbf{\textit{#1}}}}
\newcommand{\WarningTok}[1]{\textcolor[rgb]{0.56,0.35,0.01}{\textbf{\textit{#1}}}}
\newcommand{\AlertTok}[1]{\textcolor[rgb]{0.94,0.16,0.16}{#1}}
\newcommand{\ErrorTok}[1]{\textcolor[rgb]{0.64,0.00,0.00}{\textbf{#1}}}
\newcommand{\NormalTok}[1]{#1}
\usepackage{graphicx,grffile}
\makeatletter
\def\maxwidth{\ifdim\Gin@nat@width>\linewidth\linewidth\else\Gin@nat@width\fi}
\def\maxheight{\ifdim\Gin@nat@height>\textheight\textheight\else\Gin@nat@height\fi}
\makeatother
% Scale images if necessary, so that they will not overflow the page
% margins by default, and it is still possible to overwrite the defaults
% using explicit options in \includegraphics[width, height, ...]{}
\setkeys{Gin}{width=\maxwidth,height=\maxheight,keepaspectratio}
\IfFileExists{parskip.sty}{%
\usepackage{parskip}
}{% else
\setlength{\parindent}{0pt}
\setlength{\parskip}{6pt plus 2pt minus 1pt}
}
\setlength{\emergencystretch}{3em}  % prevent overfull lines
\providecommand{\tightlist}{%
  \setlength{\itemsep}{0pt}\setlength{\parskip}{0pt}}
\setcounter{secnumdepth}{0}
% Redefines (sub)paragraphs to behave more like sections
\ifx\paragraph\undefined\else
\let\oldparagraph\paragraph
\renewcommand{\paragraph}[1]{\oldparagraph{#1}\mbox{}}
\fi
\ifx\subparagraph\undefined\else
\let\oldsubparagraph\subparagraph
\renewcommand{\subparagraph}[1]{\oldsubparagraph{#1}\mbox{}}
\fi

%%% Use protect on footnotes to avoid problems with footnotes in titles
\let\rmarkdownfootnote\footnote%
\def\footnote{\protect\rmarkdownfootnote}

%%% Change title format to be more compact
\usepackage{titling}

% Create subtitle command for use in maketitle
\providecommand{\subtitle}[1]{
  \posttitle{
    \begin{center}\large#1\end{center}
    }
}

\setlength{\droptitle}{-2em}

  \title{Taller 1 Multivariado}
    \pretitle{\vspace{\droptitle}\centering\huge}
  \posttitle{\par}
    \author{Diego Torres}
    \preauthor{\centering\large\emph}
  \postauthor{\par}
      \predate{\centering\large\emph}
  \postdate{\par}
    \date{Febrero de 2020}


\begin{document}
\maketitle

\subsubsection{\texorpdfstring{\textbf{Estudio
climático}}{Estudio climático}}\label{estudio-climuxe1tico}

Se tiene una base de datos mensual entre mayo de 1983 y diciembre de
2008 con informacion relacionada a contaminación del aire a nivel
global, en donde se tienen datos de concentraciones de contaminantes,
medidas de temperatura, indices, etc.La descripción de las variables es
la siguiente:

\textbf{Year:} el año de la observacion.\\
\textbf{Month:} el mes de la observacion.\\
\textbf{Temp:} la diferencia en grados centigrados entre la temperatura
promedio global en ese periodo y el valor de referencia. Estos datos
fueron suministrados por la Unidad de Investigacion Climatica de la
University of East Anglia.\\
\textbf{CO2, N2O, CH4, CFC.11, CFC.12:} concentraciones atmosfericas de
dioxido de carbono (CO2), oxido nitroso (N2O), metano (CH4),
triclorofluorometano (CCl3F; mas conocido como CFC-11) and
diclorodifluorometano (CCl2F2; mas conocido como CFC-12),
respectivamente. CFC.11 y CFC.12 son expresadas en ppbv (partes por
billon de volumen), mientras que CO2, N2O y CH4 son expresados en ppmv
(partes por millon de volumen, es decir, 397 ppmv de CO2 significa que
ese CO2 constituye 397 millonesimas partes del volumen total de la
atmosfera). Estos datos provienen de ESRL/NOAA Division Global de
Monitoreo.\\
\textbf{Aerosols:} la profundidad óptica media del aerosol
estratosferico a 550 nm. Esta variable es asociada a la actividad
volcanica y al material piroclastico acomulado en la atmosfera, lo cual
afecta la cantidad de energía solar que es reflejada de vuelta al
espacio. Los datos fueron suministrados por el Instituto Godard de
Estudios Espaciales de la NASA.\\
\textbf{TSI:} el total de irradiacion solar (TSI) en W/m2 (la tasa a la
que la energia del sol es depositada por unidad de area). Debido a las
manchas solares y otros fenomenos similares, la cantidad de energia
emitida por el sol varia sustancialmente con el tiempo. La información
fue obtenida de el proyecto web SOLARIS-HEPPA.\\
\textbf{MEI:} indice multivariado de oscilación meridional de El Niño
(MEI), una medida que indica la fuerza de oscilación de los fenomenos de
El Nino/La Nina en el hemisferio sur (Son fenomenos climáticos que
ocurren en el pacifico sur que afecta temperaturas globales). Los datos
provienen del ESRL/NOAA División de Ciencias Fisicas.\\
A continuación, un resumen de los datos.

\begin{verbatim}
## 'data.frame':    308 obs. of  11 variables:
##  $ Year    : Factor w/ 26 levels "1983","1984",..: 1 1 1 1 1 1 1 1 2 2 ...
##  $ Month   : Factor w/ 12 levels "1","2","3","4",..: 5 6 7 8 9 10 11 12 1 2 ...
##  $ MEI     : num  2.556 2.167 1.741 1.13 0.428 ...
##  $ CO2     : num  346 346 344 342 340 ...
##  $ CH4     : num  1639 1634 1633 1631 1648 ...
##  $ N2O     : num  304 304 304 304 304 ...
##  $ CFC.11  : num  191 192 193 194 194 ...
##  $ CFC.12  : num  350 352 354 356 357 ...
##  $ TSI     : num  1366 1366 1366 1366 1366 ...
##  $ Aerosols: num  0.0863 0.0794 0.0731 0.0673 0.0619 0.0569 0.0524 0.0486 0.0451 0.0416 ...
##  $ Temp    : num  0.109 0.118 0.137 0.176 0.149 0.093 0.232 0.078 0.089 0.013 ...
## NULL
\end{verbatim}

\begin{verbatim}
##       Year         Month          MEI               CO2       
##  1984   : 12   5      : 26   Min.   :-1.6350   Min.   :340.2  
##  1985   : 12   6      : 26   1st Qu.:-0.3987   1st Qu.:353.0  
##  1986   : 12   7      : 26   Median : 0.2375   Median :361.7  
##  1987   : 12   8      : 26   Mean   : 0.2756   Mean   :363.2  
##  1988   : 12   9      : 26   3rd Qu.: 0.8305   3rd Qu.:373.5  
##  1989   : 12   10     : 26   Max.   : 3.0010   Max.   :388.5  
##  (Other):236   (Other):152                                    
##       CH4            N2O            CFC.11          CFC.12     
##  Min.   :1630   Min.   :303.7   Min.   :191.3   Min.   :350.1  
##  1st Qu.:1722   1st Qu.:308.1   1st Qu.:246.3   1st Qu.:472.4  
##  Median :1764   Median :311.5   Median :258.3   Median :528.4  
##  Mean   :1750   Mean   :312.4   Mean   :252.0   Mean   :497.5  
##  3rd Qu.:1787   3rd Qu.:317.0   3rd Qu.:267.0   3rd Qu.:540.5  
##  Max.   :1814   Max.   :322.2   Max.   :271.5   Max.   :543.8  
##                                                                
##       TSI          Aerosols            Temp        
##  Min.   :1365   Min.   :0.00160   Min.   :-0.2820  
##  1st Qu.:1366   1st Qu.:0.00280   1st Qu.: 0.1217  
##  Median :1366   Median :0.00575   Median : 0.2480  
##  Mean   :1366   Mean   :0.01666   Mean   : 0.2568  
##  3rd Qu.:1366   3rd Qu.:0.01260   3rd Qu.: 0.4073  
##  Max.   :1367   Max.   :0.14940   Max.   : 0.7390  
## 
\end{verbatim}

\begin{Shaded}
\begin{Highlighting}[]
\KeywordTok{library}\NormalTok{(ggplot2)}
\KeywordTok{ggplot}\NormalTok{(data, }\KeywordTok{aes}\NormalTok{(}\DataTypeTok{x =}\NormalTok{ Year, }\DataTypeTok{y =}\NormalTok{ Temp, }\DataTypeTok{colour =}\NormalTok{ Month)) }\OperatorTok{+}
\StringTok{  }\KeywordTok{geom_point}\NormalTok{() }\OperatorTok{+}\StringTok{ }
\StringTok{  }\KeywordTok{scale_x_discrete}\NormalTok{(}\DataTypeTok{name=}\StringTok{"Years"}\NormalTok{, }\DataTypeTok{breaks =} \KeywordTok{c}\NormalTok{(}\KeywordTok{seq}\NormalTok{(}\DecValTok{1983}\NormalTok{, }\DecValTok{2008}\NormalTok{, }\DecValTok{3}\NormalTok{))) }\OperatorTok{+}\StringTok{ }
\StringTok{  }\KeywordTok{scale_y_continuous}\NormalTok{(}\DataTypeTok{name=}\StringTok{"Temperature °C"}\NormalTok{, }\DataTypeTok{limits=}\KeywordTok{c}\NormalTok{(}\OperatorTok{-}\FloatTok{0.3}\NormalTok{, }\FloatTok{0.8}\NormalTok{)) }\OperatorTok{+}\StringTok{ }
\StringTok{  }\KeywordTok{theme}\NormalTok{(}\DataTypeTok{axis.line =} \KeywordTok{element_line}\NormalTok{(}\DataTypeTok{colour =} \StringTok{"grey"}\NormalTok{, }
                      \DataTypeTok{size =} \DecValTok{1}\NormalTok{, }\DataTypeTok{linetype =} \StringTok{"solid"}\NormalTok{)) }\OperatorTok{+}\StringTok{ }
\StringTok{  }\KeywordTok{ggtitle}\NormalTok{(}\StringTok{"Difference in Global temperature between 1983 and 2008"}\NormalTok{)}
\end{Highlighting}
\end{Shaded}

\includegraphics{Taller-1-Multivariado_files/figure-latex/unnamed-chunk-2-1.pdf}

El grafico indica una marcada tendencia de ascenso de los niveles de
temperatura mundial con respecto a un valor de referencia, los colores
son dados por el promedio mensual de cada año, mostrando que los
primeros meses del año suelen ser más calidos.Vamos a ver la insidencia
de algunos contaminantes y otros fenómenos climaticos en la prediccion
de estos ascensos de la temperatura.

Ahora, vamos a ver la relacion entre las demas variables
cualitativas.Para esto, se obtendrá la matriz de covarianza, la cual nos
permite ver la varianza de cada variable en la diagonal principal.
Mientras que los demás valores son la covarianza e indican la relación
de dependencia entre cada par de variables de la matriz. Si ambas
aumentan o disminuyen, la covarianza es positiva, mientras que si una
variable tiende a aumentar y la otra no, la covarianza sera negativa. La
covarianza es similar a la correlación, pero cuando esta ultima se
calcula, los datos están estandarizados. Por lo tanto, la covarianza se
expresa en unidades que varían con los datos y no se convierte a una
escala estandarizada de 1 negativo a +1 como sí ocurre en la
correlación.

\begin{Shaded}
\begin{Highlighting}[]
\KeywordTok{library}\NormalTok{(dplyr)}
\NormalTok{data. <-}\StringTok{ }\KeywordTok{select}\NormalTok{(data, MEI}\OperatorTok{:}\NormalTok{Temp)}
\KeywordTok{round}\NormalTok{(}\KeywordTok{cov}\NormalTok{(data.), }\DecValTok{3}\NormalTok{)}
\end{Highlighting}
\end{Shaded}

\begin{verbatim}
##             MEI     CO2      CH4     N2O  CFC.11   CFC.12    TSI Aerosols
## MEI       0.880  -1.814   -4.559  -0.796   1.673   -2.161 -0.029    0.010
## CO2      -1.814 159.950  508.019  64.836 102.678  602.050  0.090   -0.136
## CH4      -4.559 508.019 2120.757 215.218 664.777 2551.810  2.693   -0.388
## N2O      -0.796  64.836  215.218  27.302  43.570  253.596  0.083   -0.054
## CFC.11    1.673 102.678  664.777  43.570 409.325  972.667  2.301   -0.019
## CFC.12   -2.161 602.050 2551.810 253.596 972.667 3343.950  4.374   -0.410
## TSI      -0.029   0.090    2.693   0.083   2.301    4.374  0.160    0.001
## Aerosols  0.010  -0.136   -0.388  -0.054  -0.019   -0.410  0.001    0.001
## Temp      0.023   1.695    5.771   0.696   1.377    7.135  0.013   -0.002
##            Temp
## MEI       0.023
## CO2       1.695
## CH4       5.771
## N2O       0.696
## CFC.11    1.377
## CFC.12    7.135
## TSI       0.013
## Aerosols -0.002
## Temp      0.032
\end{verbatim}

Se observa de la matriz que hay una fuerte relación de dependencia entre
los contaminantes principales de la atmosfera. Para mirar un poco más la
relación de dependencia, utilizaremos un grafico de
correlación.Posteriormente, un grafico que nos indicara la distribucion
de cada variable, el valor de la correlación y si es significante o no.

\begin{Shaded}
\begin{Highlighting}[]
\KeywordTok{library}\NormalTok{(corrplot)}
\KeywordTok{corrplot}\NormalTok{(}\KeywordTok{cor}\NormalTok{(data.), }\DataTypeTok{type =} \StringTok{"upper"}\NormalTok{, }\DataTypeTok{order =} \StringTok{"hclust"}\NormalTok{, }
         \DataTypeTok{tl.col =} \StringTok{"black"}\NormalTok{, }\DataTypeTok{tl.srt =} \DecValTok{45}\NormalTok{)}
\end{Highlighting}
\end{Shaded}

\includegraphics{Taller-1-Multivariado_files/figure-latex/unnamed-chunk-4-1.pdf}

\begin{Shaded}
\begin{Highlighting}[]
\KeywordTok{library}\NormalTok{(}\StringTok{"PerformanceAnalytics"}\NormalTok{)}
\KeywordTok{chart.Correlation}\NormalTok{(data., }\DataTypeTok{histogram=}\OtherTok{TRUE}\NormalTok{, }\DataTypeTok{pch=}\DecValTok{19}\NormalTok{)}
\end{Highlighting}
\end{Shaded}

\includegraphics{Taller-1-Multivariado_files/figure-latex/unnamed-chunk-5-1.pdf}

Del ultimo grafico podemos observar que hay dos contaminantes, el CO2 y
el N2O ( correlación de 0.75 y 0.74 con respecto a temp), que son
altamente correlacionados al delta de temperatura global, además de que
parecen estar distribuidos normalmente. Veamos estas dos variables con
mas detalle.

\paragraph{\texorpdfstring{\textbf{Dioxido de carbono
CO2}}{Dioxido de carbono CO2}}\label{dioxido-de-carbono-co2}

\begin{Shaded}
\begin{Highlighting}[]
\KeywordTok{summary}\NormalTok{(data}\OperatorTok{$}\NormalTok{CO2)}
\end{Highlighting}
\end{Shaded}

\begin{verbatim}
##    Min. 1st Qu.  Median    Mean 3rd Qu.    Max. 
##   340.2   353.0   361.7   363.2   373.5   388.5
\end{verbatim}

Ahora veamos la distribución y el plot de densidad:

\begin{Shaded}
\begin{Highlighting}[]
\KeywordTok{ggplot}\NormalTok{(data, }\KeywordTok{aes}\NormalTok{(}\DataTypeTok{x =}\NormalTok{ data}\OperatorTok{$}\NormalTok{CO2)) }\OperatorTok{+}\StringTok{ }
\StringTok{  }\KeywordTok{geom_histogram}\NormalTok{(}\DataTypeTok{binwidth=}\DecValTok{4}\NormalTok{, }\DataTypeTok{color=}\StringTok{"white"}\NormalTok{, }\DataTypeTok{fill=}\StringTok{"sandybrown"}\NormalTok{, }\KeywordTok{aes}\NormalTok{(}\DataTypeTok{y=}\NormalTok{..density..)) }\OperatorTok{+}
\StringTok{  }\KeywordTok{geom_density}\NormalTok{(}\DataTypeTok{stat=}\StringTok{"density"}\NormalTok{, }\DataTypeTok{alpha=}\KeywordTok{I}\NormalTok{(}\FloatTok{0.1}\NormalTok{), }\DataTypeTok{fill=}\StringTok{"papayawhip"}\NormalTok{, }\DataTypeTok{lty=}\DecValTok{2}\NormalTok{) }\OperatorTok{+}
\StringTok{  }\KeywordTok{scale_x_continuous}\NormalTok{(}\DataTypeTok{name=}\StringTok{"CO2 ppmv"}\NormalTok{, }\DataTypeTok{breaks =} \KeywordTok{c}\NormalTok{(}\KeywordTok{seq}\NormalTok{(}\DecValTok{340}\NormalTok{, }\DecValTok{390}\NormalTok{, }\DecValTok{6}\NormalTok{))) }\OperatorTok{+}\StringTok{ }
\StringTok{  }\KeywordTok{theme}\NormalTok{(}\DataTypeTok{axis.line =} \KeywordTok{element_line}\NormalTok{(}\DataTypeTok{colour =} \StringTok{"grey"}\NormalTok{, }
                      \DataTypeTok{size =} \DecValTok{1}\NormalTok{, }\DataTypeTok{linetype =} \StringTok{"solid"}\NormalTok{)) }\OperatorTok{+}\StringTok{ }
\StringTok{  }\KeywordTok{ggtitle}\NormalTok{(}\StringTok{"Histogram & Density Curve CO2 (1983-2008)"}\NormalTok{)}
\end{Highlighting}
\end{Shaded}

\includegraphics{Taller-1-Multivariado_files/figure-latex/unnamed-chunk-7-1.pdf}

Aplicamos test de Shapiro-Wilks para determinar normalidad en los datos:

\begin{Shaded}
\begin{Highlighting}[]
\CommentTok{#Normalidad}
\KeywordTok{shapiro.test}\NormalTok{(data}\OperatorTok{$}\NormalTok{CO2) }
\end{Highlighting}
\end{Shaded}

\begin{verbatim}
## 
##  Shapiro-Wilk normality test
## 
## data:  data$CO2
## W = 0.96349, p-value = 5.401e-07
\end{verbatim}

H0: las muestras provienen de una distribución normal.\\
H1: las muestras no provienen de una distribución normal.

Según Royston (1995), el valor del p-valor es adecuado para valores
menores a 0.1, sin embargo, el test de shapiro es contra la normalidad
en los datos, por lo tanto, al ser el p-valor menor al nivel de
significancia, entonces se rechaza la hipotesis de que provienen de una
distribución normal.\\

Ahora, obtendremos el diagrama de tallo y hojas que nos ayudará a
visualizar de manera más numérica la distribución.

\begin{Shaded}
\begin{Highlighting}[]
\KeywordTok{stem}\NormalTok{(data}\OperatorTok{$}\NormalTok{CO2)}
\end{Highlighting}
\end{Shaded}

\begin{verbatim}
## 
##   The decimal point is at the |
## 
##   340 | 23457
##   342 | 311122
##   344 | 12455890355689
##   346 | 01155778012678
##   348 | 011244790127899
##   350 | 02345733445667
##   352 | 123368999123778889
##   354 | 01123399991345566779
##   356 | 1123389991334468
##   358 | 011245900113344556699
##   360 | 236788834778
##   362 | 0224022258
##   364 | 0123345599224568
##   366 | 0377990117
##   368 | 012234600335566779
##   370 | 35567802556788
##   372 | 25511224789
##   374 | 013469056799
##   376 | 2577005679
##   378 | 23557792689
##   380 | 2456891148
##   382 | 0222445790189
##   384 | 1125670467
##   386 | 0144629
##   388 | 5
\end{verbatim}

Ahora, observaremos como ha variado el dioxido de carbono a traves de
los años según los datos:

\begin{Shaded}
\begin{Highlighting}[]
\KeywordTok{ggplot}\NormalTok{(data, }\KeywordTok{aes}\NormalTok{(}\DataTypeTok{x =}\NormalTok{ data}\OperatorTok{$}\NormalTok{Year, }\DataTypeTok{y =}\NormalTok{ data}\OperatorTok{$}\NormalTok{CO2)) }\OperatorTok{+}\StringTok{ }
\StringTok{  }\KeywordTok{geom_boxplot}\NormalTok{() }\OperatorTok{+}\StringTok{ }
\StringTok{  }\KeywordTok{scale_x_discrete}\NormalTok{(}\DataTypeTok{name=}\StringTok{"Years"}\NormalTok{, }\DataTypeTok{breaks =} \KeywordTok{c}\NormalTok{(}\KeywordTok{seq}\NormalTok{(}\DecValTok{1980}\NormalTok{, }\DecValTok{2008}\NormalTok{, }\DecValTok{3}\NormalTok{))) }\OperatorTok{+}\StringTok{ }
\StringTok{  }\KeywordTok{scale_y_continuous}\NormalTok{(}\DataTypeTok{name=}\StringTok{"CO2 ppmv"}\NormalTok{) }\OperatorTok{+}\StringTok{ }
\StringTok{  }\KeywordTok{theme}\NormalTok{(}\DataTypeTok{axis.line =} \KeywordTok{element_line}\NormalTok{(}\DataTypeTok{colour =} \StringTok{"grey"}\NormalTok{, }
                      \DataTypeTok{size =} \DecValTok{1}\NormalTok{, }\DataTypeTok{linetype =} \StringTok{"solid"}\NormalTok{)) }\OperatorTok{+}\StringTok{ }
\StringTok{  }\KeywordTok{ggtitle}\NormalTok{(}\StringTok{"CO2 (1983-2008)"}\NormalTok{) }
\end{Highlighting}
\end{Shaded}

\includegraphics{Taller-1-Multivariado_files/figure-latex/unnamed-chunk-10-1.pdf}

\paragraph{\texorpdfstring{\textbf{Oxido nitroso
N2O}}{Oxido nitroso N2O}}\label{oxido-nitroso-n2o}

\begin{Shaded}
\begin{Highlighting}[]
\KeywordTok{summary}\NormalTok{(data}\OperatorTok{$}\NormalTok{N2O)}
\end{Highlighting}
\end{Shaded}

\begin{verbatim}
##    Min. 1st Qu.  Median    Mean 3rd Qu.    Max. 
##   303.7   308.1   311.5   312.4   317.0   322.2
\end{verbatim}

Ahora veamos la distribución y el plot de densidad:

\begin{Shaded}
\begin{Highlighting}[]
\KeywordTok{ggplot}\NormalTok{(data, }\KeywordTok{aes}\NormalTok{(}\DataTypeTok{x =}\NormalTok{ data}\OperatorTok{$}\NormalTok{N2O)) }\OperatorTok{+}\StringTok{ }
\StringTok{  }\KeywordTok{geom_histogram}\NormalTok{(}\DataTypeTok{binwidth=}\DecValTok{2}\NormalTok{, }\DataTypeTok{color=}\StringTok{"white"}\NormalTok{, }\DataTypeTok{fill=}\StringTok{"papayawhip"}\NormalTok{, }\KeywordTok{aes}\NormalTok{(}\DataTypeTok{y=}\NormalTok{..density..)) }\OperatorTok{+}
\StringTok{  }\KeywordTok{geom_density}\NormalTok{(}\DataTypeTok{stat=}\StringTok{"density"}\NormalTok{, }\DataTypeTok{alpha=}\KeywordTok{I}\NormalTok{(}\FloatTok{0.1}\NormalTok{), }\DataTypeTok{fill=}\StringTok{"sandybrown"}\NormalTok{, }\DataTypeTok{lty=}\DecValTok{2}\NormalTok{) }\OperatorTok{+}
\StringTok{  }\KeywordTok{scale_x_continuous}\NormalTok{(}\DataTypeTok{name=}\StringTok{"N2O ppmv"}\NormalTok{, }\DataTypeTok{breaks =} \KeywordTok{c}\NormalTok{(}\KeywordTok{seq}\NormalTok{(}\DecValTok{300}\NormalTok{, }\DecValTok{325}\NormalTok{, }\DecValTok{2}\NormalTok{))) }\OperatorTok{+}\StringTok{ }
\StringTok{  }\KeywordTok{theme}\NormalTok{(}\DataTypeTok{axis.line =} \KeywordTok{element_line}\NormalTok{(}\DataTypeTok{colour =} \StringTok{"grey"}\NormalTok{, }
                      \DataTypeTok{size =} \DecValTok{1}\NormalTok{, }\DataTypeTok{linetype =} \StringTok{"solid"}\NormalTok{)) }\OperatorTok{+}\StringTok{ }
\StringTok{  }\KeywordTok{ggtitle}\NormalTok{(}\StringTok{"Histogram & Density Curve N2O (1983-2008)"}\NormalTok{)}
\end{Highlighting}
\end{Shaded}

\includegraphics{Taller-1-Multivariado_files/figure-latex/unnamed-chunk-12-1.pdf}

Aplicamos test de Shapiro-Wilks para determinar normalidad en los datos:

\begin{Shaded}
\begin{Highlighting}[]
\CommentTok{#Normalidad}
\KeywordTok{shapiro.test}\NormalTok{(data}\OperatorTok{$}\NormalTok{N2O) }
\end{Highlighting}
\end{Shaded}

\begin{verbatim}
## 
##  Shapiro-Wilk normality test
## 
## data:  data$N2O
## W = 0.95066, p-value = 1.177e-08
\end{verbatim}

H0: las muestras provienen de una distribución normal.\\
H1: las muestras no provienen de una distribución normal.

Al ser el p-valor menor al nivel de significancia, entonces se rechaza
la hipotesis de que provienen de una distribución normal.\\

Ahora, obtendremos el diagrama de tallo y hojas que nos ayudará a
visualizar de manera más numérica la distribución.

\begin{Shaded}
\begin{Highlighting}[]
\KeywordTok{stem}\NormalTok{(data}\OperatorTok{$}\NormalTok{N2O)}
\end{Highlighting}
\end{Shaded}

\begin{verbatim}
## 
##   The decimal point is at the |
## 
##   303 | 77889
##   304 | 00112345679
##   305 | 0000112223333445578899
##   306 | 1111223444455555666667788999
##   307 | 3466677788
##   308 | 024455666777789
##   309 | 001124444455557779
##   310 | 0000111111122222222333344455778889999
##   311 | 1122345556777889
##   312 | 04455666778
##   313 | 0022334555679
##   314 | 234445566679
##   315 | 01122246899
##   316 | 00111222233357889999
##   317 | 0000122234666678
##   318 | 01111112222356889999999
##   319 | 135788899999
##   320 | 01333455556669
##   321 | 123334444458
##   322 | 02
\end{verbatim}

Ahora, observaremos como ha variado el oxido nitroso a traves de los
años según los datos:

\begin{Shaded}
\begin{Highlighting}[]
\KeywordTok{ggplot}\NormalTok{(data, }\KeywordTok{aes}\NormalTok{(}\DataTypeTok{x =}\NormalTok{ data}\OperatorTok{$}\NormalTok{Year, }\DataTypeTok{y =}\NormalTok{ data}\OperatorTok{$}\NormalTok{N2O)) }\OperatorTok{+}\StringTok{ }
\StringTok{  }\KeywordTok{geom_boxplot}\NormalTok{() }\OperatorTok{+}\StringTok{ }
\StringTok{  }\KeywordTok{scale_x_discrete}\NormalTok{(}\DataTypeTok{name=}\StringTok{"Years"}\NormalTok{, }\DataTypeTok{breaks =} \KeywordTok{c}\NormalTok{(}\KeywordTok{seq}\NormalTok{(}\DecValTok{1980}\NormalTok{, }\DecValTok{2008}\NormalTok{, }\DecValTok{3}\NormalTok{))) }\OperatorTok{+}\StringTok{ }
\StringTok{  }\KeywordTok{scale_y_continuous}\NormalTok{(}\DataTypeTok{name=}\StringTok{"N2O ppmv"}\NormalTok{) }\OperatorTok{+}\StringTok{ }
\StringTok{  }\KeywordTok{theme}\NormalTok{(}\DataTypeTok{axis.line =} \KeywordTok{element_line}\NormalTok{(}\DataTypeTok{colour =} \StringTok{"grey"}\NormalTok{, }
                      \DataTypeTok{size =} \DecValTok{1}\NormalTok{, }\DataTypeTok{linetype =} \StringTok{"solid"}\NormalTok{)) }\OperatorTok{+}\StringTok{ }
\StringTok{  }\KeywordTok{ggtitle}\NormalTok{(}\StringTok{"N2O (1983-2008)"}\NormalTok{) }
\end{Highlighting}
\end{Shaded}

\includegraphics{Taller-1-Multivariado_files/figure-latex/unnamed-chunk-15-1.pdf}

Se puede observar que la variación mensual de las concentraciones en
volumen de N2O varia poco, pero cuando se compara entre años, las
diferencias son muy significativas.

Ahora bien, para finalizar realizaremos un modelo de predicción de la
variable temperatura a partir del resto de variables.

\begin{Shaded}
\begin{Highlighting}[]
\NormalTok{model <-}\StringTok{ }\KeywordTok{lm}\NormalTok{(Temp }\OperatorTok{~}\StringTok{ }\NormalTok{., }\DataTypeTok{data =}\NormalTok{ data.)}
\KeywordTok{summary}\NormalTok{(model)}
\end{Highlighting}
\end{Shaded}

\begin{verbatim}
## 
## Call:
## lm(formula = Temp ~ ., data = data.)
## 
## Residuals:
##      Min       1Q   Median       3Q      Max 
## -0.26228 -0.05868  0.00051  0.05718  0.32170 
## 
## Coefficients:
##               Estimate Std. Error t value Pr(>|t|)    
## (Intercept) -1.277e+02  1.919e+01  -6.654 1.36e-10 ***
## MEI          6.632e-02  6.186e-03  10.722  < 2e-16 ***
## CO2          5.207e-03  2.192e-03   2.375   0.0182 *  
## CH4          6.371e-05  4.977e-04   0.128   0.8982    
## N2O         -1.693e-02  7.835e-03  -2.161   0.0315 *  
## CFC.11      -7.278e-03  1.461e-03  -4.980 1.07e-06 ***
## CFC.12       4.272e-03  8.763e-04   4.875 1.77e-06 ***
## TSI          9.586e-02  1.401e-02   6.844 4.38e-11 ***
## Aerosols    -1.582e+00  2.099e-01  -7.535 5.86e-13 ***
## ---
## Signif. codes:  0 '***' 0.001 '**' 0.01 '*' 0.05 '.' 0.1 ' ' 1
## 
## Residual standard error: 0.09182 on 299 degrees of freedom
## Multiple R-squared:  0.744,  Adjusted R-squared:  0.7371 
## F-statistic: 108.6 on 8 and 299 DF,  p-value: < 2.2e-16
\end{verbatim}

Del modelo anterior, obtenemos que las variables contaminantes como CO2,
CH4 y N2O son las menos significativas, quizas porque la información que
estas aportan al modelo es redundante, debido a su alta correlación.

\begin{Shaded}
\begin{Highlighting}[]
\NormalTok{model <-}\StringTok{ }\KeywordTok{lm}\NormalTok{(Temp }\OperatorTok{~}\StringTok{ }\NormalTok{.}\OperatorTok{-}\NormalTok{N2O}\OperatorTok{-}\NormalTok{CH4}\OperatorTok{-}\NormalTok{CO2, }\DataTypeTok{data =}\NormalTok{ data.)}
\KeywordTok{summary}\NormalTok{(model)}
\end{Highlighting}
\end{Shaded}

\begin{verbatim}
## 
## Call:
## lm(formula = Temp ~ . - N2O - CH4 - CO2, data = data.)
## 
## Residuals:
##      Min       1Q   Median       3Q      Max 
## -0.26695 -0.06084  0.00119  0.05607  0.32378 
## 
## Coefficients:
##               Estimate Std. Error t value Pr(>|t|)    
## (Intercept) -1.272e+02  1.913e+01  -6.651 1.36e-10 ***
## MEI          6.777e-02  6.132e-03  11.052  < 2e-16 ***
## CFC.11      -6.200e-03  5.104e-04 -12.147  < 2e-16 ***
## CFC.12       3.656e-03  1.791e-04  20.412  < 2e-16 ***
## TSI          9.313e-02  1.402e-02   6.644 1.42e-10 ***
## Aerosols    -1.660e+00  2.086e-01  -7.960 3.50e-14 ***
## ---
## Signif. codes:  0 '***' 0.001 '**' 0.01 '*' 0.05 '.' 0.1 ' ' 1
## 
## Residual standard error: 0.09235 on 302 degrees of freedom
## Multiple R-squared:  0.7385, Adjusted R-squared:  0.7341 
## F-statistic: 170.5 on 5 and 302 DF,  p-value: < 2.2e-16
\end{verbatim}

Así, por medio del método backward o de eliminación de variables, se
obtuvo un modelo mas significativo, ya que tiene mayor variabilidad.

Así, como conclusión, podemos decir que la temperatura no ha dejado de
ascender en los ultimos 30 años, y todo se debe a mayores
concentraciones de contaminantes en el aire. Además, la mayoria de
contaminantes ha tenido un aumento de 30ppmv durante el tiempo de
registro de datos, lo cual representa un aumento del casi 10\%. El
planeta necesita un cambio en las políticas ambientales e industriales
practicamente ahora, los resultados ya los podemos evidenciar.


\end{document}
